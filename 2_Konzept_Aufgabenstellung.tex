
Die Optimierung des Stromverbrauchs wird immer wichtiger. Dadurch kann die Umwelt
gesch�tzt und der eigene Energieverbrauch gesenkt werden. Ersteres wird aufgrund
der immer weiter steigenden Umweltverschmutzung bei der Energiegewinnung wichtiger.\\

Zur Reduzierung der Umweltverschmutzung bei der Energiegewinnung und damit auch
beim Energieverbrauch kann beigetragen werden, indem der eigene Verbrauch analysiert,
unn�tige Verbr�uche ermittelt und optimiert werden k�nnen. Die Reduzierung des
Energieverbrauchs hat auch einen Ausgaben reduzierenden Charakter.\\

Dazu soll in dieser Bachelorarbeit eine Monitorringl�sung entwickelt werden, die
�ber vier S0- Schnittstellen, den Energieverbrauch in Geb�uden automatisch am
Z�hler misst, verarbeitet und abspeichert. In dieser Arbeit soll mit dem Aufsatz
SD0, direkt �ber seinen vier S0-Kan�len, von den am Stromz�hler angebrachten
S0-Schnittstellen der Energieverbrauch gemessen werden. Die Abspeicherung der
Verbrauchswerte soll sowohl lokal, als auch auf einem Server erfolgen k�nnen.
Dazu gibt der Aufsatz die Verbrauchswerte an den Rasperry Pi weiter, wo die
Verbrauchswerte abgespeichert werden. Vom Raspberry Pi aus k�nnen, wenn n�tig,
die Verbrauchsdaten an einen Server weitergeleitet und dort abgelegt zu werden.
Die Visualisierung der Verbrauchswerte soll �ber eine Webdarstellung vorgenommen
werden.\\

Die Verbrauchsmessung und deren Darstellung erm�glicht, dass die Verbrauchsdaten
analysiert und der Stromverbrauch des Nutzers bzw. des Systems optimiert werden k�nnen.\\

Die aufgenommenen S0-Impulse k�nnen Werte von Elektro-, Wasser- oder Gasz�hlern
repr�sentieren. F�r die M�glichkeit der Konfiguration der Impulskonstante und der
Messleitung ist eine Einstellm�glichkeit herauszusuchen. Die entwickelte L�sung
ist prototypisch zu implementieren und in einem Testaufbau zu �berpr�fen.
