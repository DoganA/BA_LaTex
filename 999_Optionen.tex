\definecolor{light-gray}{gray}{0.95}
%%%%%%%%%%%%%%%%%%%%%%%%%%%%%%%%%%%%%%%%%%%%%%%%%%%%%%%%%%%%%%%%%%%%%%%%%%%%%%%
%%Seitenlayout definieren
\geometry{a4paper,
  left  = 25mm,
  right = 20mm,
  top   = 2.5cm,
  bottom= 3cm,
}

%%%%%%%%%%%%%%%%%%%%%%%%%%%%%%%%%%%%%%%%%%%%%%%%%%%%%%%%%%%%%%%%%%%%%%%%%%%%%%%
%% Quellenformatierung für Tabellen und Grafiken
\newcommand*{\quelle}[1]{%
	\footnotesize(\uline{Quelle:} #1)
}

%%%%%%%%%%%%%%%%%%%%%%%%%%%%%%%%%%%%%%%%%%%%%%%%%%%%%%%%%%%%%%%%%%%%%%%%%%%%%%%
%% Quellenformatierung für Tabellen und Grafiken
\newcommand*{\tableRowHeader}[1]{%
	\cellcolor{lightgray}\textbf{\underline{#1}}
}
%%%%%%%%%%%%%%%%%%%%%%%%%%%%%%%%%%%%%%%%%%%%%%%%%%%%%%%%%%%%%%%%%%%%%%%%%%%%%%%
%% Fuß- und Kopfzeile bei Kapitelseiten:
\renewcommand{\chapterpagestyle}{scrheadings}

%%%%%%%%%%%%%%%%%%%%%%%%%%%%%%%%%%%%%%%%%%%%%%%%%%%%%%%%%%%%%%%%%%%%%%%%%%%%%%%
%% Untersteichen der Chapter und subchapter
%\RedeclareSectionCommands[font=\normalsize]{chapter,section,subsection,subsubsection}

\renewcommand*{\chapterlinesformat}[3]{%
  #2\uline{#3}%
}
\renewcommand*{\sectionlinesformat}[4]{%
  \hskip#2#3\uline{#4}% 
}

%%%%%%%%%%%%%%%%%%%%%%%%%%%%%%%%%%%%%%%%%%%%%%%%%%%%%%%%%%%%%%%%%%%%%%%%%%%%%%%
%%% Tabellen und Abbildungs Verzeichniss formatierung: Es iwrd so formatier, dass
%%% Prfäx Fettgedruckt und getrent zum Text mit einem : ist (Quelle: https://komascript.de/node/1911)
\addtokomafont{captionlabel}{\bfseries} %nötig, für fettdruck
\addtokomafont{captionlabel}{\uline} %nötig, für unterstrich

%Grafik
\renewcommand*\listoflofentryname{\bfseries\figurename}
	%\renewcaptionname{ngerman}{\figurename}{Abb.}
\BeforeStartingTOC[lof]{\renewcommand*\autodot{:}}

%Tabellen
\renewcommand*\listoflotentryname{\bfseries\tablename}
\BeforeStartingTOC[lot]{\renewcommand*\autodot{:}}

%%%%%%%%%%%%%%%%%%%%%%%%%%%%%%%%%%%%%%%%%%%%%%%%%%%%%%%%%%%%%%%%%%%%%%%%%%%%%%%
%%%% Im Text Tab. und Abb. Captions werden Fett und unterstrichenn ausgegeben
\makeatletter
\renewcommand\caption@@make[2]{%
  \centering\underline{\textbf{#1:}}~#2}
\makeatother
%%%%%%%%%%%%%%%%%%%%%%%%%%%%%%%%%%%%%%%%%%%%%%%%%%%%%%%%%%%%%%%%%%%%%%%%%%%%%%%
%%%Bibliographie jede 2. Eintrag Hintergrundfarbe.
\newcounter{zebrabibentry}
\newcounter{zebrabibbibenv}

\defbibenvironment{bibliography}
  {\list
     {\stepcounter{zebrabibentry}
      \zebrabibstart{zebrabib-%
        \the\value{zebrabibbibenv}-%
        \the\value{zebrabibentry}}%
      \printtext[labelnumberwidth]{%
        \printfield{labelprefix}%
        \printfield{labelnumber}}}
     {\stepcounter{zebrabibbibenv}%
      \setcounter{zebrabibentry}{0}%
      \setlength{\labelwidth}{\labelnumberwidth}%
      \setlength{\leftmargin}{\labelwidth}%
      \setlength{\labelsep}{\biblabelsep}%
      \addtolength{\leftmargin}{\labelsep}%
      \setlength{\itemsep}{\bibitemsep}%
      \setlength{\parsep}{\bibparsep}}%
      \renewcommand*{\makelabel}[1]{\hss##1}}
  {\endlist}
  {\item}

\newcommand*{\zebrabibstart}[2][]{%
  \tikz[remember picture,overlay]
  \draw[line width=1pt,rectangle,
    draw=\ifodd\value{zebrabibentry}white\else blue!20\fi,
    fill=\ifodd\value{zebrabibentry}white\else blue!10\fi,]
    let \p1=(pic cs:#2) in
    ({0pt,10pt}) node [anchor=base] (#2){} rectangle  (\columnwidth+2pt,\y1-\itemsep)
    ;
}

\newcommand\zebrabibend[2][]{%
  \tikz[remember picture with id=#2] #1;}

\renewbibmacro{finentry}{%
  \finentry
  \zebrabibend{zebrabib-%
    \the\value{zebrabibbibenv}-%
    \the\value{zebrabibentry}}}
%%%%%%%%%%%%%%%%%%%%%%%%%%%%%%%%%%%%%%%%%%%%%%%%%%%%%%%%%%%%%%%%%%%%%%%%%%%%%%%
%%%Tabellenformatierung
%Tabellenverzeichnis
\lstset{
    frame = LL, 										% draw a frame at the top and bottom of the code block
    tabsize = 4, 									% tab space width
    showstringspaces = false, 	% don't mark spaces in strings
    numbers=left, 							% display line numbers on the left
    commentstyle=\color{green}, % comment color
    keywordstyle=\color{blue}, 	% keyword color
    stringstyle=\color{red} 		% string color
}
\lstdefinelanguage{json}{
    string=[s]{"}{"},
    stringstyle=\color{blue},
    comment=[l]{:},
    commentstyle=\color{black},
}
