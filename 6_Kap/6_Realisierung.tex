In diesem Kapitel wird die Realisierung der prototypischen Entwicklung beschrieben.

Um die Verbrauchswerte zu messen gibt es schon ein Projekt, den Volksz�hler. Da-bei handelt es sich um ein Open Source Projekt [vgl. vza]. Mit dem Volksz�hler k�n-nen Energieverbrauchswerte gemessen und auf dessen Darstellung Verbrauchswer-te von Wasser, Strom Gas usw. angezeigt werden [vgl. Zoe], mit dem die Nutzer die eigenen Verbrauchdaten wie Strom, Temperatur, Wasser am Z�hler messen k�nnen. Die gemessenen Verbrauchsdaten werden dann �ber die Schnittstellen an das Messger�t �bertragen, dort gespeichert und auf einer Website ausgewertet. Ein ge-eignetes Messger�t muss vorhanden sein. In dieser Arbeit handelt es sich dabei um den SD0 von Busware, zusammen mit dem Raspberry Pi. Die Messung findet dabei am Z�hler �ber die S0-Schnittstelle statt. Die �bertragung findet �ber Kupferkabel statt. Die Speicherung und Auswertung finden dann durch die Software statt. Bei der Software handelt es sich um den Volksz�hler. Mit dem Volksz�hler k�nnen auch Verbrauchsdaten �ber S0-Schnittstellen gemessen werden [vgl. vzg]. Dazu gibt es zwei Softwares, um S0-Impulse zu messen. Dabei handelt es sich um s0vz [vgl. s0vz] und s0enow. Aus dem Quellcode bzw. der Datei s0enow.cpp, in der sich auch die Main-Funktion befindet, geht hervor, dass sich beim Projekt s0enow um einen nach GNU General Public Licence der Version 3 Lizenzierte Software handelt. Das bedeutet, dass der Quellcode verwendet bzw. Modifiziert werden darf. Dabei muss aber die Software, die Teile der Software, die nach GNU General Public Licence der Version 3 lizenziert wurde, verwendet auch als solche lizenziert werden [vgl. s0ec Codezeilen 23 bis 38] \cite[vgl.][S. 455]{s0ec}.

In dieser Arbeit werden einige Funktionen, die im Softwareprojekt s0enow vorhanden sind, verwendet. Dadurch m�ssen diese nicht von neuem entwickelt werden. Neben der Bibliothek, die zum Einlesen der Konfigurationsdatei dient, werden �hnliche Da-ten eingelesen, wie im Projekt s0enow. Dies ist der Tatsache geschuldet, dass die Software die in dieser Arbeit entwickelt wird, eine �hnliche Aufgabe erf�llt, wie das Projekt s0enow. Dabei handelt es sich um die Funktion cfile(). Diese Datei befindet sich in den Codezeilen 204 bis 309 im s0enow Projekt [vgl. s0ec].
In der Software, die in dieser Arbeit entwickelt wird, werden neben dem Datafolder, die Messstellenname, die Mittelwertszeit auch die Impulskonstanten ausgelesen.

Zus�tzlich gibt es �berschneidungen mit den Projekten s0enow und s0vz. Wie in diesen Projekten wird die Software, die auf dem Raspberry Pi laufen soll, als Dienst-programm bzw. Deamon entwickelt So genannte Deamons laufen unter Linux im Hintergrund und starten, wenn das Betriebssystem hochgefahren wird [vgl. Wol2006 dpz und Wol2006 af]. Diese Funktionen befinden sich in den Codezeilen 111 bis 202, im Projekt s0enow [vgl. s0ec]. Bei den Funktionen handelt es sich um die drei fol-genden Funktionen,
