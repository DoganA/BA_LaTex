Energie ist f�r vieles n�tig. Ohne Energie w�ren die bisherige industrielle Entwicklung
und das Leben wie es bekannt ist, nicht m�glich. Erst Energietr�ger wie �l, Erdgas und
vor allem elektrische Energie sollen den wirtschaftlichen und technischen Fortschritt
erst erm�glichen haben \cite[vgl.][S. 4 - S. 6]{Bpb2013} and \cite[vgl.][S. 4 - S. 6]{Bpb2013}. 2014 wurden 684 TWh Strom
erzeugt. Die Quelle der erzeugten elektrischen Energien wurden vor allem durch
fossile Brennstoffe erzeugt \cite[vgl.][S.38]{EDG}.

Aufgrund des Energieverbrauchs finden auch Klima�nderungen durch Umweltverschmutzung,
verursacht durch Treibhausgase wie Kohlendioxid, statt. Dies hat nicht nur eine
steigende globale Durchschnittstemperatur, sondern auch extreme Wetterereignisse,
wie St�rme und �berflutungen zur Folge \cite[vgl.][]{Ven2013}. Die globale Durchschnittstemperatur
soll um bis zu vier Grad Celsius steigen, wenn nichts dagegen unternommen wird.

Die Reduzierung des Stromverbrauchs hat nicht nur klimasch�tzende, sondern auch
Kosten reduzierende Wirkung, da verminderte Nachfrage und Verbrauch auch entsprechend
niedrigere Ausgaben f�r Strom bedeuten. Die Kontrolle des eigenen Energieverbrauchs
wird daher immer wichtiger. Damit ist nicht nur die �berwachung der eigenen Energiekosten
m�glich, sondern auch die Steigerung der Energieeffizienz von Anlagen \cite[vgl.][]{pae}.
Zudem wird durch die Energieverbrauchskontrolle die M�glichkeit er�ffnet, die
Umwelt zu schonen. Die Reduzierung des eigenen Stromverbrauchs und die Vermeidung
von Spitzen sorgen daf�r, dass die Stromversorgung und somit auch die Zulieferung
und Produktion von Strom einged�mmt wird. Umweltschutz wird also nicht nur dadurch
gef�rdert, wie und mit welchen Mitteln Strom gewonnen wird, z.B. durch erneuerbare
Energien, sondern auch durch die Nachfrage danach. Um zum Umweltschutz beizutragen
kommt es also nicht nur dar-auf an, wie der Strom erzeugt wird, sondern auch wie viel
Energie verbraucht wird. Es h�ngt somit auch vom eigenen Verbrauch ab, weil die Menge
des produzierten Stroms vermindert wird. Da die Stromerzeugung auch von der Nachfrage
abh�ngig ist.

Hier kommt die in dieser Arbeit entwickelte Softwarel�sung zum Einsatz. Mit der
S0-Schnittstelle, nach DIN EN 62053-31, kann nicht nur der aktuelle Verbrauchstand
von den Z�hlern direkt abgelesen, sondern auch �ber ganze Zeitr�ume hinweg erfasst
werden, die dann aufbereitet und Visualisiert werden. So werden die Verbrauchsdaten
nicht nur f�r einen bestimmten Zeitpunkt dargestellt, sondern f�r ganze Zeitr�ume.

Bei der Visualisierung der Energieverbrauchsdaten �ber die Zeit k�nnen die Verbrauchsdaten
analysiert werden. Durch die Analyse k�nnen unn�tige Verbr�uche erfasst, dargestellt,
eventuelle Zusammenh�ngen erkannt und so der Energieverbrauch optimiert werden.