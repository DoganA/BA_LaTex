In dieser Abschlussarbeit wurde eine, aus zwei Bereichen bestehende, Softwarel�sung
entwickelt, die den eigenen Energieverbrauch visualisiert und dadurch die M�glichkeit
aufgreift den eigenen Energieverbrauch zu analysieren. Die auf der Grundlage eines
Linux-Betriebssystems entwickelte Software, welche auf einem Raspberry Pi l�uft,
wertet die empfangenen S0-Impulse in Energieverbrauchswerte um. Auf der Grundlage,
von diesen als Linuxtreiber (auch Deamon bezeichnet) entwickelten Software
ausgewerteten Verbrauchswerten, findet die Visualisierung, der ausgewerteten
Energieverbrauchswerte statt. Die Visualisierungseinheit, also die Webseite befindet
sich dann auf einem Webserver, der entweder auf einem Raspberry Pi oder einem vom
Raspberry Pi physisch unabh�ngigen Server l�uft. Die entwickelte Visualisierungseinheit
bietet den Vorteil, dass nicht nur der bisherige Gesamtverbrauch, vom Z�hler als
Zahl abgelesen werden kann, sondern die Verbrauchswerte f�r einen bestimmten
Zeitraum grafisch darzustellen werden. Zudem ist es durch die Auswahl der einzelnen
Kan�len, die der Anzahl der S0-Anschl�ssen der Erweiterung SD0 entsprechen, m�glich
die Verbrauchswerte verschiedener Z�hler, gleichzeitig zu analysieren und miteinander
zu vergleichen. Es wurde eine Anforderungsanalyse durchgef�hrt, die die Anforderungen
an die entwickelte Softwarel�sungen definieren soll. Mit deren Hilfe wurden dann
im anschlie�enden Kapitel die L�sungen entworfen. Ausgehend vom Entwurf, der die
Grundlage bzw. den Ausgangspunkt der realisierten L�sungen bildet, wurde die L�sung
realisiert. Es wurde auch ein Parser zum Einlesen von Konfigurationsdateien verwendet.
Dadurch kann die Software w�hrend der Laufzeit an verschiedene Z�hler angepasst
werden. Die Software, die auf dem Raspberry Pi l�uft, wurde modular aufgebaut.
Dadurch ist sp�ter eine einfache Erweiterung der schon entwickelten Software m�glich.

Der Vorteil bei der in dieser Abschlussarbeit eingesetzten S0-Schnittstelle ist,
dass die Schnittstelle nicht nur f�r den Energieverbrauch, sondern auch f�r andere
Verbrauchswerte wie W�rme, Wasser und Gas verwendet werden kann. Die entwickelte
Software kann also in Verbindung mit den S0-Schnittstellen vielseitig verwendet
werden.

Eine M�glichkeit den Prototyp zu erweitern w�re, weitere Darstellungsformen und 
-varianten hinzuzuf�gen. Das w�re beispielsweise die Darstellung der Verbrauchsdaten
in einer App, also auf einem Smartphone oder Tablet. Es w�re auch m�glich, weitere
Daten, wie Tarifinformationen und bisherige Kosten f�r den Energie-verbrauch,
darzustellen. Es w�re auch m�glich, die aktuellen Energieverbrauchswerte mit den
Verbrauchswerten des selben Zeitraums vergangener Jahre zu vergleichen.