
Bei der Datenhaltung, also der Datenspeicherung, gibt es verschiedene M�glichkeiten:
die Daten k�nnen beispielsweise als CSV, also als Comma Separated Values,
abgespeichert werden. Es gibt jedoch auch die M�glichkeiten die Daten in
JSON oder aber auch in SQL abzulegen. Die Daten in SQL abzuspeichern lohnt sich
erst, wenn die Daten auch lokal auf dem Raspberry Pi weiterverarbeitet werden sollen.
Dabei muss auch bedacht werden, dass ein SQL-Server ben�tigt wird. Dies
w�rde dazu f�hren, dass durch den SQL-Server mehr Ressourcen verwenden w�rden.
Das hei�t ein SQL-Server w�rde nicht nur Speicher auf der SD-Karte einnehmen,
sondern auch, wenn der Server l�uft, den Arbeitsspeicher und die CPU beanspruchen.

Beim JSON ist bei der Programmierung des Quellcodes, die auf dem Raspberry Pi
l�uft, nur ein Libary n�tig, der die Daten in eine. JSON-Datei schreibt. JSON wird
meist im Web Bereich eingesetzt.


\subsection{Comma Separated Values (CSV)}
\label{sec:CommaSeparatedValuesCSV}
emeinen
Standard oder bestimmte Spezifikation, sondern es kann in unterschiedlichen
Spezifikationen eingesetzt werden. Das CSV-Format kann dann eingesetzt,
wenn die abgespeicherten Daten unabh�ngig von der Technologie eingesetzt werden
sollen. Also wenn die Daten beispielsweise in einem Tabellenprogramm dargestellt
und in einem Quellcode bearbeitet werden sollen \cite[vgl.]{RFC4180}.

Die Daten k�nnen entweder durch einfache Kommata voneinander getrennt in der
.csv Datei abgelegt werden, die Werte an sich k�nnen zus�tzlich in Anf�hrungsstriche
bzw -zeichen gesetzt werden. Genauso k�nnten die einzelnen Daten mit Semikolon
voneinander getrennt abgespeichert werden \cite[vgl.]{RFC4180}. Zu beachten ist
jedoch, dass bei Kommazahlen im deutschen ein Komma benutzt wird und im englischsprachigen
Raum ein Punkt. Dies k�nnte evtl. zu Problemen beim Lesen bzw.
Parsen der Datei im Quellcode f�hren.


\subsection{JavaScript Object Notation (JSON)}
\label{sec:JavaScriptObjectNotationJSON}
JavaScript Object Notation, kurz JSON, ist ein Datenaustauschformat. Dabei handelt
es sich um ein von Programmiersprachen unabh�ngiges Textformat. Daten k�nnen
also programmiersprachenunabh�ngig ausgetauscht werden. JSON folgt jedoch
der JavaScript Notation. Dabei handelt es sich bei JSON um eine Untermenge der
Skriptsprache JavaScript \cite[vgl.]{ecma2013}.

Das Datenaustauschformat basiert dabei auf zwei Strukturen \cite[vgl.]{ecma2013}:
\begin{description}
	\item \textbf{Name-Wert-Paare}\\	
	Dabei ist jedem Namen bzw. Bezeichnung mindestens ein Wert zugewiesen.
	
	\item \textbf{Geordnete Liste von Werten}\\
Neben den beiden Strukturen gibt es in JSON, wie auch in Programmiersprachen, Typen. Die Typen beziehen sich auf die Werte. Die folgende Auflistung mit entsprechenden Zeichen
[vgl ecma2013]:

	\item \textbf{Objekte}\\
\textit{Objekte} beinhaltet eine ungeordnete Menge an Name-Werte-Paare. Es beginnt mit einer �ffnenden geschweiften Klammer "`{"` und einer schlie�enden geschweiften Klammer "`}"'. Die Name-Werte-Paare sind mit einem Doppelpunkt getrennt [vgl. ecma2013]:\\
\item \textit{Name:Wert}\\
Jedes Paar ist wiederum zueinander mit Kommata (,) voneinander getrennt [vgl. ecma2013].

	\item \textbf{Arrays}\\
Ein Array beinhaltet eine geordnete Liste von Werten, die einem Namen zugewiesen werden. Arrays beginnen mit einer [ (�ffnenden eckigen Klammer) und enden mit einer ] (schlie�enden eckigen Klammer). Die Werte an sich sind mit , (Komma) zu einander getrennt [vgl ecma2013].

\item \textbf{Werte}\\
Werte werden durch Objekte, Arrays, Zeichenketten (Strings), Zahlen, den Wahrheitswerten \textsl{true} und false oder Null gebildet [vgl ecma2013].
\item \textbf{Zeichenketten}\\
Strings, also Zeichenketten, sind Unicode Zeichen [vgl ecma2013].

\item \textbf{Nummern}\\
Bei den Zahlen in JSON handelt es sich um Dezimalzahlen. Den Zahlen kann ein -
(Minus) vorangestellt werden. Gefolgt von entweder einem 0 (Null) oder einer beliebigen Zahl von 1 bis 9 in beliebiger Weiderholung. Gleitkommerzahlen werden mit einem Punkt getrennt [vgl ecma2013]. \cite{ecma2013}
\end{description}
Im Folgenden ist ein beispielhafter Aufbau einer JSON-Datei dargestellt:

\newpage
\begin{lstlisting}[language=json, caption={Beispielhafter Aufbau einer JSON-Datei}]
{
	"`books"': [
		"`book"': {
			"`language"':"'Java"',
			"`edition"':"'second"',
			"`price"': 10,
			"`author"': ["`Hans"', "`Peter"', "`M�ller"']
		},

		"`book"': {
			"`language"':"'C++"',
			"`lastName"':"'fifth"'
			"`price"': 11.12,
			"`author"': ["`Stroustroup"']
		},

		"`book"': {
			"`language"':"'C"',
			"`lastName"':"'third"'
			"`price"': 8,
			"`author"': ["`Georg"', "`Petrus"']
		}
	]
}
\end{lstlisting}
\begin{center}
\quelle{\cite[vgl.][]{asd}}
\end{center}